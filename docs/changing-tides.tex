% Options for packages loaded elsewhere
\PassOptionsToPackage{unicode}{hyperref}
\PassOptionsToPackage{hyphens}{url}
%
\documentclass[
]{book}
\usepackage{amsmath,amssymb}
\usepackage{lmodern}
\usepackage{iftex}
\ifPDFTeX
  \usepackage[T1]{fontenc}
  \usepackage[utf8]{inputenc}
  \usepackage{textcomp} % provide euro and other symbols
\else % if luatex or xetex
  \usepackage{unicode-math}
  \defaultfontfeatures{Scale=MatchLowercase}
  \defaultfontfeatures[\rmfamily]{Ligatures=TeX,Scale=1}
\fi
% Use upquote if available, for straight quotes in verbatim environments
\IfFileExists{upquote.sty}{\usepackage{upquote}}{}
\IfFileExists{microtype.sty}{% use microtype if available
  \usepackage[]{microtype}
  \UseMicrotypeSet[protrusion]{basicmath} % disable protrusion for tt fonts
}{}
\makeatletter
\@ifundefined{KOMAClassName}{% if non-KOMA class
  \IfFileExists{parskip.sty}{%
    \usepackage{parskip}
  }{% else
    \setlength{\parindent}{0pt}
    \setlength{\parskip}{6pt plus 2pt minus 1pt}}
}{% if KOMA class
  \KOMAoptions{parskip=half}}
\makeatother
\usepackage{xcolor}
\usepackage{longtable,booktabs,array}
\usepackage{calc} % for calculating minipage widths
% Correct order of tables after \paragraph or \subparagraph
\usepackage{etoolbox}
\makeatletter
\patchcmd\longtable{\par}{\if@noskipsec\mbox{}\fi\par}{}{}
\makeatother
% Allow footnotes in longtable head/foot
\IfFileExists{footnotehyper.sty}{\usepackage{footnotehyper}}{\usepackage{footnote}}
\makesavenoteenv{longtable}
\usepackage{graphicx}
\makeatletter
\def\maxwidth{\ifdim\Gin@nat@width>\linewidth\linewidth\else\Gin@nat@width\fi}
\def\maxheight{\ifdim\Gin@nat@height>\textheight\textheight\else\Gin@nat@height\fi}
\makeatother
% Scale images if necessary, so that they will not overflow the page
% margins by default, and it is still possible to overwrite the defaults
% using explicit options in \includegraphics[width, height, ...]{}
\setkeys{Gin}{width=\maxwidth,height=\maxheight,keepaspectratio}
% Set default figure placement to htbp
\makeatletter
\def\fps@figure{htbp}
\makeatother
\setlength{\emergencystretch}{3em} % prevent overfull lines
\providecommand{\tightlist}{%
  \setlength{\itemsep}{0pt}\setlength{\parskip}{0pt}}
\setcounter{secnumdepth}{5}
\usepackage{booktabs}
\usepackage{amsthm}
\makeatletter
\def\thm@space@setup{%
  \thm@preskip=8pt plus 2pt minus 4pt
  \thm@postskip=\thm@preskip
}
\makeatother
\ifLuaTeX
  \usepackage{selnolig}  % disable illegal ligatures
\fi
\usepackage[]{natbib}
\bibliographystyle{apalike}
\IfFileExists{bookmark.sty}{\usepackage{bookmark}}{\usepackage{hyperref}}
\IfFileExists{xurl.sty}{\usepackage{xurl}}{} % add URL line breaks if available
\urlstyle{same} % disable monospaced font for URLs
\hypersetup{
  pdftitle={Dòng Triều Đổi Thay (Changing Tides)},
  pdfauthor={(alphabetical order): Celestine Le, Rachel Moran, Anh Nguyen, and Sarah Nguyễn},
  hidelinks,
  pdfcreator={LaTeX via pandoc}}

\title{Dòng Triều Đổi Thay (Changing Tides)}
\author{(alphabetical order): Celestine Le, Rachel Moran, Anh Nguyen, and Sarah Nguyễn}
\date{2023-03-14}

\begin{document}
\maketitle

{
\setcounter{tocdepth}{1}
\tableofcontents
}
\hypertarget{duxf2ng-triux1ec1u-ux111ux1ed5i-thay-changing-tides}{%
\chapter{Dòng Triều Đổi Thay (Changing Tides)}\label{duxf2ng-triux1ec1u-ux111ux1ed5i-thay-changing-tides}}

This is the digital version of Dòng Triều Đổi Thay (Changing Tides), a bilingual illustrated booklet (aka zine).

\begin{center}\rule{0.5\linewidth}{0.5pt}\end{center}

(Tiếng Việt)

Mai Phùng rất hài lòng với cuộc sống của mình. Cô chỉ quan tâm tới hai điều: hiệu thuốc và gia đình.

Nhưng khi một nhà hoạt động trẻ tuổi đưa cô một tờ rơi nói rằng tòa nhà cộng động sắp phải đóng cửa, Mai và cháu trai của cô có những bất đồng ý kiến khi bàn về việc bảo vệ nơi này.

Giữa dòng chảy thông tin không ngừng, Mai phải tự hỏi rằng mình có thể dựa vào ai nhiều nhất khi truy tìm sự thật. Cùng lúc đó, cô sẽ tìm lại được một điều về bản thân mà cô tưởng rằng mình đã quên.

\begin{center}\rule{0.5\linewidth}{0.5pt}\end{center}

(English)

Mai Phung is very content with her life. She only cares about two things: running her local pharmacy and providing for her family.

But when a young activist hands her a leaflet about saving the local community building from its impending closure, Mai and her young nephew are divided over whether the building should be saved.

Caught in the ebbing currents of information, Mai must ask herself whom she can rely on the most in her search for the truth, and along the way, she rediscovers a truth about herself she had long forgotten.

\hypertarget{cuxe1ch-ux111ux1ecdc-tux1ea1p-chuxed-nuxe0y-how-to-read-this-zine}{%
\chapter{Cách đọc tạp chí này (How to read this zine)}\label{cuxe1ch-ux111ux1ecdc-tux1ea1p-chuxed-nuxe0y-how-to-read-this-zine}}

Cuốn sách này có tiếng Việt (trang 4-19) và tiếng Anh (trang 22-37).

Cuốn sách này là một câu chuyện với Kết Thúc Tự Chọn. Điều này có nghĩa là từ trang 15-19 (tiếng Việt) và trang 33-37 (tiếng Anh) bạn có thể quyết định số phận của các nhân vật và xem cái kết mình mong muốn. Ví dụ, nếu bạn chọn kết thúc \#2 bạn sẽ phải sang trang 17 để đọc kết thúc tương ứng. Chúng tôi mong bạn sẽ xem lại từng cái kết để xem những hướng khác nhau mà câu chuyện có thể đi tới.

Câu chuyện này được dựa trên nghiên cứu khoa học. Vui lòng chia sẻ những suy nghĩ, phản hồi, và kinh nghiệm của bạn tại \url{https://bit.ly/viet-zine-survey}

Tác giả (theo thứ tự bảng chữ cái): Celestine Le, Rachel Moran, Anh Nguyen, và Sarah Nguyễn
Hoạ sĩ minh hoạ: Anh Nguyen
Dịch giả: Trung-Anh Nguyen
Biên tập viên: Nguyên Lê, Hoàng Kim Nguyễn, và Kenney Tran

\begin{center}\rule{0.5\linewidth}{0.5pt}\end{center}

This book available in both Vietnamese (pages 4 - 19) and English (pages 22-37).

This book is a Choose-Your-Own-Adventure story. This means from pages 13-18 (Vietnamese) and pages 33-37 (English) you can decide the characters' fates and choose which ending you want to experience. For example, if you choose ending \#2 you will have to flip to page 35 to read the corresponding ending. We encourage you to revisit each ending to see the different ways the story could go.

This story is based off of empirical research. Please Share your reflections, feedback, and experiences at \url{https://bit.ly/viet-zine-survey}

Authors (alphabetical order): Celestine Le, Rachel Moran, Anh Nguyen, and Sarah Nguyễn
Graphic artist: Anh Nguyen
Translator: Trung-Anh Nguyen
Copy editor: Nguyên Lê, Hoàng Kim Nguyễn, and Kenney Tran

\hypertarget{duxf2ng-triux1ec1u-ux111ux1ed5i-thay}{%
\chapter{Dòng Triều Đổi Thay}\label{duxf2ng-triux1ec1u-ux111ux1ed5i-thay}}

\hypertarget{changing-tides}{%
\chapter{Changing Tides}\label{changing-tides}}

\hypertarget{thay-ux111ux1ed5i-quuxe1-khuxf3}{%
\chapter{Thay đổi quá khó}\label{thay-ux111ux1ed5i-quuxe1-khuxf3}}

\hypertarget{cuxf4-muux1ed1n-biux1ebft-thuxeam}{%
\chapter{Cô muốn biết thêm}\label{cuxf4-muux1ed1n-biux1ebft-thuxeam}}

\hypertarget{ux1eeba-ux111ux1ea3o-vuxe0-thao-tuxfang-muxe0-thuxf4i}{%
\chapter{ừa đảo và thao túng mà thôi}\label{ux1eeba-ux111ux1ea3o-vuxe0-thao-tuxfang-muxe0-thuxf4i}}

\hypertarget{ux111ux1ea7u-tux1b0-vuxe0o-cux1ed9ng-ux111ux1ed3ng}{%
\chapter{đầu tư vào cộng đồng}\label{ux111ux1ea7u-tux1b0-vuxe0o-cux1ed9ng-ux111ux1ed3ng}}

\hypertarget{change-is-too-hard}{%
\chapter{Change is too hard}\label{change-is-too-hard}}

\hypertarget{i-want-to-hear-more}{%
\chapter{I want to hear more}\label{i-want-to-hear-more}}

\hypertarget{fraud-and-manipulation}{%
\chapter{Fraud and manipulation}\label{fraud-and-manipulation}}

\hypertarget{invest-in-the-community}{%
\chapter{Invest in the community}\label{invest-in-the-community}}

\hypertarget{cux1ea3m-ux1a1n-thank-you}{%
\chapter{Cảm ơn // Thank you}\label{cux1ea3m-ux1a1n-thank-you}}

Cảm ơn vì đã đọc tạp chí này!

Câu chuyện hư cấu này được dựa trên nghiên cứu khoa học cho dự án nghiên cứu ``Gửi Tin Tức Về Nhà: Phân Tích Sự Lây Lan của Chứng Rối Loạn Thông Tin trong Cộng Đồng Người Việt Hải Ngoại trong Kỳ Bầu Cử Tổng Thống Mỹ 2020''. Đây là một nghiên cứu định tính đa phương pháp sử dụng dữ liệu từ Facebook trong khoảng thời gian có Kỳ Bầu Cử Tổng Thống Mỹ 2020 và từ thảo luận nhóm tập trung với người Mỹ gốc Việt trong năm 2021. Từ nghiên cứu này chúng tôi đã hiểu rõ hơn mạng lưới thông tin và truyền thông mà người Mỹ gốc Việt chia sẻ qua nhiều thế hệ.

Tất cả các nhân vật và hội thoại trong tạp chí này đều dựa trên kết quả nghiên cứu của chúng tôi, nhưng tất cả các tên, địa điểm, và sự kiện cụ đều không dựa trên sự thật.

Để biết thêm chi tiết về nghiên cứu này, xin đón đọc các bài báo học thuật được xuất bản tại CSCW 2023 và tạp chí Truyền Thông Chính Trị.

Tạp chí này được đến với quý vị nhờ vào sự tài trợ từ Viện Dữ Liệu, Dân Chủ, và Chính Trị tại Đại Học George Washington và Trung Tâm vì một Công Chúng Hiểu Biết tại Đại Học Washington. Lời cảm ơn đặc biệt tới sự hỗ trợ từ Việt Kiểm Tin, Người Thông Dịch, và cộng đồng người Mỹ gốc Việt vì đã sẵn sàng cộng tác và ủng hộ nghiên cứu mà câu chuyện này dựa trên.

Dòng Triều Đổi Thay CC-BY-SA-NC 2023

Xuất bản tại Seattle, WA

\begin{center}\rule{0.5\linewidth}{0.5pt}\end{center}

Thank you for reading this Zine!

This fictional story is based on empirical research conducted for the research project titled ``Sending News Back Home: Analyzing the Spread of Information Disorder in the Vietnamese Diaspora during the 2020 US Presidential Election''. This is a multi-method qualitative research study that draws on Facebook data from around the 2020 U.S. Presidential Elections and focus group discussions with Vietnamese Americans in 2021. Through this research we came to better understand the information and communication networks that Vietnamese Americans share across generations.

All characters and dialog in this zine are based on findings from our research, but all names, places, and specific events are fictionalized.

For more details about this research, keep an eye out for scholarly articles published in CSCW 2023 and the journal of Political Communication.

This zine was brought to you with funding from the George Washington University's Institute for Data, Democracy \& Politics and University of Washington's Center for an Informed Public. Special thanks to support from Viet Fact Check, The Interpreter, and the Vietnamese American community who has been open to collaboration and supportive of the research which this story is based on.

Changing Tides CC-BY-SA-NC 2023

Published in Seattle, WA

  \bibliography{book.bib,packages.bib}

\end{document}
